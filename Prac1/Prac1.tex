% Options for packages loaded elsewhere
\PassOptionsToPackage{unicode}{hyperref}
\PassOptionsToPackage{hyphens}{url}
%
\documentclass[
]{article}
\usepackage{amsmath,amssymb}
\usepackage{lmodern}
\usepackage{iftex}
\ifPDFTeX
  \usepackage[T1]{fontenc}
  \usepackage[utf8]{inputenc}
  \usepackage{textcomp} % provide euro and other symbols
\else % if luatex or xetex
  \usepackage{unicode-math}
  \defaultfontfeatures{Scale=MatchLowercase}
  \defaultfontfeatures[\rmfamily]{Ligatures=TeX,Scale=1}
  \setmainfont[]{SourceSansPro}
\fi
% Use upquote if available, for straight quotes in verbatim environments
\IfFileExists{upquote.sty}{\usepackage{upquote}}{}
\IfFileExists{microtype.sty}{% use microtype if available
  \usepackage[]{microtype}
  \UseMicrotypeSet[protrusion]{basicmath} % disable protrusion for tt fonts
}{}
\makeatletter
\@ifundefined{KOMAClassName}{% if non-KOMA class
  \IfFileExists{parskip.sty}{%
    \usepackage{parskip}
  }{% else
    \setlength{\parindent}{0pt}
    \setlength{\parskip}{6pt plus 2pt minus 1pt}}
}{% if KOMA class
  \KOMAoptions{parskip=half}}
\makeatother
\usepackage{xcolor}
\usepackage[margin=1in]{geometry}
\usepackage{graphicx}
\makeatletter
\def\maxwidth{\ifdim\Gin@nat@width>\linewidth\linewidth\else\Gin@nat@width\fi}
\def\maxheight{\ifdim\Gin@nat@height>\textheight\textheight\else\Gin@nat@height\fi}
\makeatother
% Scale images if necessary, so that they will not overflow the page
% margins by default, and it is still possible to overwrite the defaults
% using explicit options in \includegraphics[width, height, ...]{}
\setkeys{Gin}{width=\maxwidth,height=\maxheight,keepaspectratio}
% Set default figure placement to htbp
\makeatletter
\def\fps@figure{htbp}
\makeatother
\setlength{\emergencystretch}{3em} % prevent overfull lines
\providecommand{\tightlist}{%
  \setlength{\itemsep}{0pt}\setlength{\parskip}{0pt}}
\setcounter{secnumdepth}{-\maxdimen} % remove section numbering
\ifLuaTeX
  \usepackage{selnolig}  % disable illegal ligatures
\fi
\IfFileExists{bookmark.sty}{\usepackage{bookmark}}{\usepackage{hyperref}}
\IfFileExists{xurl.sty}{\usepackage{xurl}}{} % add URL line breaks if available
\urlstyle{same} % disable monospaced font for URLs
\hypersetup{
  pdftitle={Практическая работа №1. Первичная обработка данных},
  pdfauthor={Юрченков Иван Александрович, ассистент кафедры ПМ},
  hidelinks,
  pdfcreator={LaTeX via pandoc}}

\title{Практическая работа №1. Первичная обработка данных}
\author{Юрченков Иван Александрович, ассистент кафедры ПМ}
\date{2022-10-22}

\begin{document}
\maketitle

\hypertarget{ux432ux432ux435ux434ux435ux43dux438ux435}{%
\section{\texorpdfstring{\textbf{Введение}}{Введение}}\label{ux432ux432ux435ux434ux435ux43dux438ux435}}

Для прикладных задач математической статистики базовым элементом анализа
статистических данных является процедура первичного исследования
имеющихся выборок. В процедуру первичного исследования входят такие
техники, как построение вариационных таблиц для оценки эмпирического
распределения данных, оценка выборочных статистик, оценка вариабельности
данных и построение графиков гистограмм выборочных переменных для
визуального анализа на принадлежность выборки к одному из теоретических
распределений, на наличие выбросов и характерных паттернов. Данные
задачи решаются не в отрыве от устоявшейся процедуры статистического
анализа данных, а наборот являются базой для проведения других более
сложных статистических процедур, позволяющих нам получать ответы на
важные вопросы на основе данных.

Современные задачи статистики выполняются, в основном, с применением
компьютерных вычислительных средств, языков программирования и
статистических пакетов для выполнения сложных процедур обработки данных
над всё большим их числом. Тенденция к росту объема накопленной
статистической информации подогревает интерес к высокопроизводительным
компьютерным вычислениям, так что без современных инструментов базовые
задачи анализа выборки чаще всего уже не решаются.

Однако при большом засилии статистических программных продуктов и сред
разработки конвейеров обработки больших данных, за которыми скрыты
преднаписанные алгоритмы, концепции, приемы и техники проведения
первичного анализа данных важны для изучения и полного освоения будущими
специалистами данной области. Получение полного испчерпывающего знания о
практических приложениях обработки статистических данных и способов
расчета и проведения статистических процедур позволяет двигаться дальше
в направлении изучения науки о данных с большей скоростью и пониманием
за счет взаимосвязи большинства методов и концепций на более высоком
уровне.

\hypertarget{ux446ux435ux43bux44c-ux440ux430ux431ux43eux442ux44b}{%
\subsection{\texorpdfstring{\textbf{Цель
работы}}{Цель работы}}\label{ux446ux435ux43bux44c-ux440ux430ux431ux43eux442ux44b}}

Данная практическая работа ставит перед собой задачу познакомить с
процессом вычисления стандартных описательных статистик для выборок
данных раазличных типов и научить использовать методы агрегации
статистических данных с целью получения новых знаний об их эмпирическом
распределении.

\hypertarget{ux43fux43eux441ux442ux430ux43dux43eux432ux43aux430-ux437ux430ux434ux430ux447ux438}{%
\section{\texorpdfstring{\textbf{Постановка
задачи}}{Постановка задачи}}\label{ux43fux43eux441ux442ux430ux43dux43eux432ux43aux430-ux437ux430ux434ux430ux447ux438}}

Для выполнения задачи 1 раздела необходимо разбиться на две подгруппы.

Студенты первой подгруппы должны собрать со всей учебной группы данные о
росте в сантиметрах и данные о месяцах рождения.

Студенты второй подгруппы должны собрать со всей учебной группы данные о
росте в сантиметрах и данные о загаданном случайном целом числе на
интервале {[}0; 8{]}.

Для целочисленных данных необходимо:

\begin{enumerate}
\def\labelenumi{\arabic{enumi}.}
\item
  Построить вариационный ряд распределения абсолютных и относительных
  частот появления событий по выборке дискретных данных.
\item
  Построить полигон относительных частот для событий вариационного ряда.
\item
  Вычислить эмпирическую функцию распределения и построить её график.
\item
  Рассчитать выборочные описательные статистики:
\end{enumerate}

\begin{itemize}
\tightlist
\item
  среднее \(\ \overline{x}\ \);
\item
  математическое ожидание \(\ \mu_x\);
\item
  дисперсию \(\ D_{x}\ \);
\item
  стандартное отклонение \(\ \sigma_{x}\);
\item
  среднеквадратическое отклонение \(\ \hat{\sigma}_x\);
\item
  медиану зафиксированной выборки \(\ m_x\);
\item
  первый и третий квартиль \(\ \tau_{x,\ 0.25},\ \tau_{x,\ 0.75}\);
\item
  межквартильный размах \(\ IQR_x\);
\item
  коэффициент вариации \(\ \nu_{x}\).
\end{itemize}

Для вещественных данных необходимо:

\begin{enumerate}
\def\labelenumi{\arabic{enumi}.}
\item
  Рассчитать число групп \(\ g\ \), необходимых для квантования исходных
  данных по правилу Стёрджесса.
\item
  Вычислить значения границ групп \(\ Z_i, i=0, 1, \dots, g\ \) для
  значений выборки по правилу фиксированной величины интервала.
\item
  Построить вариационный ряд для выборки интервальных данных.
\item
  Построить гистограмму распределения относительных частот для
  рассчитанных интервалов выборки.
\item
  Вычислить эмпирическую функцию распределения и построить её график.
\item
  Рассчитать выборочные описательные статистики:
\end{enumerate}

\begin{itemize}
\tightlist
\item
  среднее \(\ \overline{x}\ \);
\item
  математическое ожидание \(\ \mu_x\);
\item
  дисперсию c использованием выборочного среднего \(\ D_{x}\ \);
\item
  стандартное отклонение \(\ \sigma_{x}\);
\item
  среднеквадратическое отклонение \(\ \hat{\sigma}_x\);
\item
  медиану зафиксированной выборки \(\ m_x\);
\item
  первый и третий квартиль \(\ \tau_{x,\ 0.25},\ \tau_{x,\ 0.75}\);
\item
  межквартильный размах \(\ IQR_x\);
\item
  коэффициент вариации \(\ \nu_{x}\).
\end{itemize}

\hypertarget{ux43fux440ux438ux43cux435ux440-ux440ux430ux441ux447ux435ux442ux430}{%
\section{\texorpdfstring{\textbf{Пример
расчета}}{Пример расчета}}\label{ux43fux440ux438ux43cux435ux440-ux440ux430ux441ux447ux435ux442ux430}}

На рассмотрение выносится набор данных или выборка с двумя переменными,
целочисленного \(X_1\) и вещественного \(X_2\) типа:

\[
\begin{pmatrix}
X_1\\
x_{11} \\
x_{12} \\
x_{13} \\
\vdots \\
x_{1n}
\end{pmatrix} 
\begin{pmatrix}
X_2\\
x_{21} \\
x_{22} \\
x_{23} \\
\vdots \\
x_{2n}
\end{pmatrix}
\]

Выборочное среднее целочисленных данных рассчитывается по соотношению:

\[
\overline{x} = \frac{1}{n} \sum\limits_{i=1}^{n} x_i.
\]

Математическое ожидание целочисленных данных рассчитывается по
определению математического ожидания дискретного ряда:

\[
\mu_x = \sum\limits_{i=1}^{g} x_i \cdot p_i.
\]

Дисперсия целочисленных данных рассчитывается по определению дисперсии
дискретного ряда с использованием выборочного среднего:

\[
D_x = \sum\limits_{i=1}^{g}\left( \zeta_i - \overline{x}  \right)^2 \cdot p_i.
\]

Стандартное отклонение с использованием дисперсии:

\[
\sigma_x = \sqrt{D_x} = \sqrt{\sum\limits_{i=1}^{g}\left( \zeta_i - \overline{x}  \right)^2 \cdot p_i}.
\]

Среднеквадратическое отклонение:

\[
\hat{\sigma}_x = \sqrt{\frac{1}{n-1} \cdot \sum\limits_{i=1}^{n}(\overline{x} - x_i)^2}.
\]

Медиана выборки \(-\) серединное значение отсортированной выборки:

\[
[x] = sort(x),\quad 
m_x = \left\{
\begin{matrix}
[x]_{n/2}, & \ n\  - \text{ нечетное}, \\ 
\left([x]_{\lfloor n/2 \rfloor} + [x]_{\lceil n/2 \rceil}\right)/2, & \ n\  - \text{ четное}.
\end{matrix}
\right.
\]

\hypertarget{ux432ux43eux43fux440ux43eux441ux44b-ux43dux430-ux437ux430ux449ux438ux442ux443-ux43fux440ux430ux43aux442ux438ux447ux435ux441ux43aux43eux439-ux440ux430ux431ux43eux442ux44b}{%
\section{\texorpdfstring{\textbf{Вопросы на защиту практической
работы}}{Вопросы на защиту практической работы}}\label{ux432ux43eux43fux440ux43eux441ux44b-ux43dux430-ux437ux430ux449ux438ux442ux443-ux43fux440ux430ux43aux442ux438ux447ux435ux441ux43aux43eux439-ux440ux430ux431ux43eux442ux44b}}

\end{document}
