% Options for packages loaded elsewhere
\PassOptionsToPackage{unicode}{hyperref}
\PassOptionsToPackage{hyphens}{url}
%
\documentclass[
]{article}
\usepackage{amsmath,amssymb}
\usepackage{lmodern}
\usepackage{iftex}
\ifPDFTeX
  \usepackage[T1]{fontenc}
  \usepackage[utf8]{inputenc}
  \usepackage{textcomp} % provide euro and other symbols
\else % if luatex or xetex
  \usepackage{unicode-math}
  \defaultfontfeatures{Scale=MatchLowercase}
  \defaultfontfeatures[\rmfamily]{Ligatures=TeX,Scale=1}
  \setmainfont[]{SourceSansPro}
\fi
% Use upquote if available, for straight quotes in verbatim environments
\IfFileExists{upquote.sty}{\usepackage{upquote}}{}
\IfFileExists{microtype.sty}{% use microtype if available
  \usepackage[]{microtype}
  \UseMicrotypeSet[protrusion]{basicmath} % disable protrusion for tt fonts
}{}
\makeatletter
\@ifundefined{KOMAClassName}{% if non-KOMA class
  \IfFileExists{parskip.sty}{%
    \usepackage{parskip}
  }{% else
    \setlength{\parindent}{0pt}
    \setlength{\parskip}{6pt plus 2pt minus 1pt}}
}{% if KOMA class
  \KOMAoptions{parskip=half}}
\makeatother
\usepackage{xcolor}
\usepackage[margin=1in]{geometry}
\usepackage{longtable,booktabs,array}
\usepackage{calc} % for calculating minipage widths
% Correct order of tables after \paragraph or \subparagraph
\usepackage{etoolbox}
\makeatletter
\patchcmd\longtable{\par}{\if@noskipsec\mbox{}\fi\par}{}{}
\makeatother
% Allow footnotes in longtable head/foot
\IfFileExists{footnotehyper.sty}{\usepackage{footnotehyper}}{\usepackage{footnote}}
\makesavenoteenv{longtable}
\usepackage{graphicx}
\makeatletter
\def\maxwidth{\ifdim\Gin@nat@width>\linewidth\linewidth\else\Gin@nat@width\fi}
\def\maxheight{\ifdim\Gin@nat@height>\textheight\textheight\else\Gin@nat@height\fi}
\makeatother
% Scale images if necessary, so that they will not overflow the page
% margins by default, and it is still possible to overwrite the defaults
% using explicit options in \includegraphics[width, height, ...]{}
\setkeys{Gin}{width=\maxwidth,height=\maxheight,keepaspectratio}
% Set default figure placement to htbp
\makeatletter
\def\fps@figure{htbp}
\makeatother
\setlength{\emergencystretch}{3em} % prevent overfull lines
\providecommand{\tightlist}{%
  \setlength{\itemsep}{0pt}\setlength{\parskip}{0pt}}
\setcounter{secnumdepth}{-\maxdimen} % remove section numbering
\ifLuaTeX
  \usepackage{selnolig}  % disable illegal ligatures
\fi
\IfFileExists{bookmark.sty}{\usepackage{bookmark}}{\usepackage{hyperref}}
\IfFileExists{xurl.sty}{\usepackage{xurl}}{} % add URL line breaks if available
\urlstyle{same} % disable monospaced font for URLs
\hypersetup{
  pdftitle={Практическая работа №5. Линейная регрессия. Оценка адекватности модели, оценка доверительных интервалов параметров.},
  pdfauthor={Юрченков Иван Александрович, ассистент кафедры ПМ},
  hidelinks,
  pdfcreator={LaTeX via pandoc}}

\title{Практическая работа №5. Линейная регрессия. Оценка адекватности
модели, оценка доверительных интервалов параметров.}
\author{Юрченков Иван Александрович, ассистент кафедры ПМ}
\date{2022-10-10}

\begin{document}
\maketitle

\begin{verbatim}
## 
## Присоединяю пакет: 'dplyr'
\end{verbatim}

\begin{verbatim}
## Следующие объекты скрыты от 'package:stats':
## 
##     filter, lag
\end{verbatim}

\begin{verbatim}
## Следующие объекты скрыты от 'package:base':
## 
##     intersect, setdiff, setequal, union
\end{verbatim}

\hypertarget{ux43fux43eux441ux442ux430ux43dux43eux432ux43aux430-ux437ux430ux434ux430ux447ux438-ux434ux43bux44f-ux432ux44bux43fux43eux43bux43dux435ux43dux438ux44f-ux43fux440ux430ux43aux442ux438ux447ux435ux441ux43aux43eux439-ux440ux430ux431ux43eux442ux44b}{%
\section{\texorpdfstring{\textbf{Постановка задачи для выполнения
практической
работы}}{Постановка задачи для выполнения практической работы}}\label{ux43fux43eux441ux442ux430ux43dux43eux432ux43aux430-ux437ux430ux434ux430ux447ux438-ux434ux43bux44f-ux432ux44bux43fux43eux43bux43dux435ux43dux438ux44f-ux43fux440ux430ux43aux442ux438ux447ux435ux441ux43aux43eux439-ux440ux430ux431ux43eux442ux44b}}

Для выполнения практического задания необходимо:

\begin{enumerate}
\def\labelenumi{\arabic{enumi}.}
\item
  Открыть папку, соотвествующую своей группе.
\item
  Открыть папку с вариантом, совпадающим с вашим номером в списке.
\end{enumerate}

В папке 3 файла с данными.

\begin{enumerate}
\def\labelenumi{\arabic{enumi}.}
\tightlist
\item
  1-ый файл содержит 2 ряда данных. Первый стоблец \$~x\$ содержит
  факторную переменную, второй столбец \$~y\$ результирующую. Для
  первого файла необходимо:
\end{enumerate}

\begin{itemize}
\item
  Оценить коэффициент корреляции Пирсона \$~r(x, y)\$ между двумя
  переменными в первом и втором столбце.
\item
  По шкале Чеддока оценить хакактеристику корреляционной связи между
  величинами.
\item
  Проверить статистическую значимость коэффициента корреляции Пирсона с
  помощью \(t\)-статистики.
\item
  Построить доверительный интервал для \$~r(x, y)\$ с надежностью
  \(\ \gamma\ = \ 0.95\).
\item
  Построить линейную регрессию между столбцами, оценить значение
  коэффициентов линейной зависимости.
\item
  Оценить значимость полученных коэффициентов прямой.
\item
  Построить доверительные интервалы для полученных коэффициентов.
\item
  Оценить адекватность модели \emph{по коэффициенту детерминации}.
\item
  Оценить интервал прогноза для линейной модели на \$~ft = 3\$ значения
  вперед.
\end{itemize}

\begin{enumerate}
\def\labelenumi{\arabic{enumi}.}
\setcounter{enumi}{1}
\tightlist
\item
  2-ой файл содержит 4 ряда данных. Первый ряд (столбец) содержит
  количественную факторную переменную, следующие два - качественную
  факторную переменную, последний - результирующую переменную. Для
  второго файла данных необходимо:
\end{enumerate}

\begin{itemize}
\item
  Необходимо с помощью теста Чоу обосновать необходимость деления
  выборки по одной из качественных факторных переменных.
\item
  Произвести разбиение и построить две линейных регрессии, оценить
  коэффициенты моделей.
\end{itemize}

\begin{enumerate}
\def\labelenumi{\arabic{enumi}.}
\setcounter{enumi}{2}
\tightlist
\item
  3-ий файл содержит 2 ряда данных. Для третьего файла данных
  необходимо:
\end{enumerate}

\begin{itemize}
\item
  Необходимо двумя способами (тест Спирмена и тест Гольдфельда-Квандта)
  определить, присутствует ли в данных гетероскедастичность.
\item
  Построить линейную регрессию, оценить значения коэффициентов модели.
\item
  Оценить значимость полученных коэффициентов и адекватность модели.
\item
  Все расчеты проводить для уровня значимости \(\alpha = 0.05\).
\end{itemize}

\hypertarget{ux43fux440ux438ux43cux435ux440-ux43fux440ux43eux432ux435ux434ux435ux43dux438ux44f-ux440ux435ux433ux440ux435ux441ux441ux438ux43eux43dux43dux43eux433ux43e-ux430ux43dux430ux43bux438ux437ux430-ux434ux43bux44f-ux440ux44fux434ux430-ux434ux430ux43dux43dux44bux445}{%
\section{\texorpdfstring{\textbf{Пример проведения регрессионного
анализа для ряда
данных}}{Пример проведения регрессионного анализа для ряда данных}}\label{ux43fux440ux438ux43cux435ux440-ux43fux440ux43eux432ux435ux434ux435ux43dux438ux44f-ux440ux435ux433ux440ux435ux441ux441ux438ux43eux43dux43dux43eux433ux43e-ux430ux43dux430ux43bux438ux437ux430-ux434ux43bux44f-ux440ux44fux434ux430-ux434ux430ux43dux43dux44bux445}}

\hypertarget{ux438ux441ux441ux43bux435ux434ux443ux435ux43cux44bux439-ux440ux44fux434-ux434ux430ux43dux43dux44bux445}{%
\subsection{\texorpdfstring{\textbf{Исследуемый ряд
данных}}{Исследуемый ряд данных}}\label{ux438ux441ux441ux43bux435ux434ux443ux435ux43cux44bux439-ux440ux44fux434-ux434ux430ux43dux43dux44bux445}}

Рассмотрим таблицу переменных парных данных \((x, y)\) одинаковой длины
без пропущенных значений для данных о цене алмазов (diamonds) с
категориальными параметрами: \(cut = Ideal\) (огранка), \(color = J\)
(цвет), \(clarity = SI2\) (чистота).

\begin{verbatim}
## [1] "x"
\end{verbatim}

\begin{verbatim}
## 1 -1.171 
##  2 0.02 
##  3 0 
##  4 0 
##  5 0.077 
##  6 0.049 
##  7 0.01 
##  8 0.039 
##  9 0.01 
##  10 0.058 
##  11 0.01 
##  12 0.01 
##  13 0.03 
##  14 0.03 
##  15 0.104 
##  16 0.104 
##  17 0.01 
##  18 0.104 
##  19 0.049 
##  20 0.095 
##  21 0.01 
##  22 0.131 
##  23 0.095 
##  24 0.113 
##  25 0.03 
##  26 0.049 
##  27 0.122 
##  28 0.174 
##  29 0.182 
##  30 0.182 
##  31 0.095 
##  32 0.231 
##  33 0.182 
##  34 0.215 
##  35 0.199 
##  36 0.239 
##  37 0.231 
##  38 0.182 
##  39 0.182 
##  40 0.239 
##  41 0.215 
##  42 0.231 
##  43 0.239 
##  44 0.239 
##  45 0.086 
##  46 0.207 
##  47 0.207 
##  48 0.293 
##  49 0.293 
##  50 0.322 
##  51 0.445 
##  52 0.322 
##  53 0.419 
##  54 0.315 
##  55 0.507 
##  56 0.438 
##  57 0.438 
##  58 0.464 
##  59 0.531 
##  60 0.536 
##  61 0.571 
##  62 0.531 
##  63 0.698 
##  64 0.723 
##  65 0.703 
##  66 0.723 
##  67 0.708 
##  68 0.732 
##  69 0.698 
##  70 0.718 
##  71 0.708 
##  72 0.703 
##  73 0.728 
##  74 0.698 
##  75 0.829 
##  76 0.698 
##  77 0.798 
##  78 0.829 
##  79 0.829 
##  80 0.916 
##  81 0.916 
##  82 0.904 
##  83 0.916 
##  84 0.928 
##  85 1.102 
##  86 0.9 
##  87 0.967 
##  88 0.959 
##  89 1.001 
##  90 0.956 
##  91 -1.109 
##  92 -0.892 
##  93 -0.942 
##  94 -0.635 
##  95 -0.673 
##  96 -0.654 
##  97 -0.616 
##  98 -0.635 
##  99 -0.462 
##  100 -0.357 
##  101 -0.357 
##  102 -0.357 
##  103 -0.274 
##  104 -0.342 
##  105 -0.329 
##  106 -0.288 
##  107 -0.357 
##  108 -0.211 
##  109 0.02 
##  110 0
\end{verbatim}

\begin{verbatim}
## [1] "y"
\end{verbatim}

\begin{verbatim}
## 1 5.841 
##  2 7.965 
##  3 8.15 
##  4 8.168 
##  5 8.171 
##  6 8.193 
##  7 8.225 
##  8 8.233 
##  9 8.243 
##  10 8.277 
##  11 8.29 
##  12 8.293 
##  13 8.296 
##  14 8.307 
##  15 8.312 
##  16 8.317 
##  17 8.318 
##  18 8.319 
##  19 8.325 
##  20 8.333 
##  21 8.337 
##  22 8.346 
##  23 8.349 
##  24 8.372 
##  25 8.377 
##  26 8.381 
##  27 8.383 
##  28 8.414 
##  29 8.414 
##  30 8.415 
##  31 8.439 
##  32 8.446 
##  33 8.447 
##  34 8.448 
##  35 8.45 
##  36 8.454 
##  37 8.464 
##  38 8.465 
##  39 8.465 
##  40 8.472 
##  41 8.473 
##  42 8.489 
##  43 8.503 
##  44 8.521 
##  45 8.524 
##  46 8.534 
##  47 8.548 
##  48 8.57 
##  49 8.586 
##  50 8.66 
##  51 8.66 
##  52 8.694 
##  53 8.714 
##  54 8.715 
##  55 8.76 
##  56 8.825 
##  57 8.849 
##  58 8.876 
##  59 8.918 
##  60 8.948 
##  61 8.958 
##  62 9.048 
##  63 9.307 
##  64 9.327 
##  65 9.334 
##  66 9.336 
##  67 9.389 
##  68 9.407 
##  69 9.439 
##  70 9.446 
##  71 9.451 
##  72 9.452 
##  73 9.455 
##  74 9.458 
##  75 9.488 
##  76 9.492 
##  77 9.525 
##  78 9.527 
##  79 9.582 
##  80 9.582 
##  81 9.632 
##  82 9.644 
##  83 9.68 
##  84 9.68 
##  85 9.683 
##  86 9.709 
##  87 9.736 
##  88 9.753 
##  89 9.787 
##  90 9.818 
##  91 5.903 
##  92 6.594 
##  93 6.111 
##  94 6.752 
##  95 6.786 
##  96 6.829 
##  97 6.886 
##  98 6.91 
##  99 6.971 
##  100 7.477 
##  101 7.51 
##  102 7.513 
##  103 7.514 
##  104 7.55 
##  105 7.563 
##  106 7.573 
##  107 7.624 
##  108 7.65 
##  109 7.867 
##  110 7.885
\end{verbatim}

\begin{longtable}[]{@{}
  >{\raggedright\arraybackslash}p{(\columnwidth - 22\tabcolsep) * \real{0.0580}}
  >{\raggedright\arraybackslash}p{(\columnwidth - 22\tabcolsep) * \real{0.1159}}
  >{\raggedright\arraybackslash}p{(\columnwidth - 22\tabcolsep) * \real{0.0870}}
  >{\raggedright\arraybackslash}p{(\columnwidth - 22\tabcolsep) * \real{0.0580}}
  >{\raggedright\arraybackslash}p{(\columnwidth - 22\tabcolsep) * \real{0.1014}}
  >{\raggedright\arraybackslash}p{(\columnwidth - 22\tabcolsep) * \real{0.0870}}
  >{\raggedright\arraybackslash}p{(\columnwidth - 22\tabcolsep) * \real{0.0580}}
  >{\raggedright\arraybackslash}p{(\columnwidth - 22\tabcolsep) * \real{0.1014}}
  >{\raggedright\arraybackslash}p{(\columnwidth - 22\tabcolsep) * \real{0.0870}}
  >{\raggedright\arraybackslash}p{(\columnwidth - 22\tabcolsep) * \real{0.0580}}
  >{\raggedright\arraybackslash}p{(\columnwidth - 22\tabcolsep) * \real{0.1014}}
  >{\raggedright\arraybackslash}p{(\columnwidth - 22\tabcolsep) * \real{0.0870}}@{}}
\caption{Таблица данных}\tabularnewline
\toprule()
\begin{minipage}[b]{\linewidth}\raggedright
n
\end{minipage} & \begin{minipage}[b]{\linewidth}\raggedright
x
\end{minipage} & \begin{minipage}[b]{\linewidth}\raggedright
y
\end{minipage} & \begin{minipage}[b]{\linewidth}\raggedright
n
\end{minipage} & \begin{minipage}[b]{\linewidth}\raggedright
x
\end{minipage} & \begin{minipage}[b]{\linewidth}\raggedright
y
\end{minipage} & \begin{minipage}[b]{\linewidth}\raggedright
n
\end{minipage} & \begin{minipage}[b]{\linewidth}\raggedright
x
\end{minipage} & \begin{minipage}[b]{\linewidth}\raggedright
y
\end{minipage} & \begin{minipage}[b]{\linewidth}\raggedright
n
\end{minipage} & \begin{minipage}[b]{\linewidth}\raggedright
x
\end{minipage} & \begin{minipage}[b]{\linewidth}\raggedright
y
\end{minipage} \\
\midrule()
\endfirsthead
\toprule()
\begin{minipage}[b]{\linewidth}\raggedright
n
\end{minipage} & \begin{minipage}[b]{\linewidth}\raggedright
x
\end{minipage} & \begin{minipage}[b]{\linewidth}\raggedright
y
\end{minipage} & \begin{minipage}[b]{\linewidth}\raggedright
n
\end{minipage} & \begin{minipage}[b]{\linewidth}\raggedright
x
\end{minipage} & \begin{minipage}[b]{\linewidth}\raggedright
y
\end{minipage} & \begin{minipage}[b]{\linewidth}\raggedright
n
\end{minipage} & \begin{minipage}[b]{\linewidth}\raggedright
x
\end{minipage} & \begin{minipage}[b]{\linewidth}\raggedright
y
\end{minipage} & \begin{minipage}[b]{\linewidth}\raggedright
n
\end{minipage} & \begin{minipage}[b]{\linewidth}\raggedright
x
\end{minipage} & \begin{minipage}[b]{\linewidth}\raggedright
y
\end{minipage} \\
\midrule()
\endhead
1 & -1.171 & 5.841 & 31 & 0.095 & 8.439 & 61 & 0.571 & 8.958 & 91 &
-1.109 & 5.903 \\
2 & 0.020 & & 32 & 0.231 & & 62 & 0.531 & & 92 & & \\
3 & 0.000 & & 33 & 0.182 & & 63 & 0.698 & & 93 & & \\
4 & 0.000 & & 34 & 0.215 & & 64 & 0.723 & & 94 & & \\
5 & 0.077 & & 35 & 0.199 & & 65 & 0.703 & & 95 & & \\
6 & 0.049 & & 36 & 0.239 & & 66 & 0.723 & & 96 & & \\
7 & 0.010 & & 37 & 0.231 & & 67 & 0.708 & & 97 & & \\
8 & 0.039 & & 38 & 0.182 & & 68 & 0.732 & & 98 & & \\
9 & 0.010 & & 39 & 0.182 & & 69 & 0.698 & & 99 & & \\
10 & 0.058 & & 40 & 0.239 & & 70 & 0.718 & & 100 & & \\
11 & 0.010 & & 41 & 0.215 & & 71 & 0.708 & & 101 & & \\
12 & 0.010 & & 42 & 0.231 & & 72 & 0.703 & & 102 & & \\
13 & 0.030 & & 43 & 0.239 & & 73 & 0.728 & & 103 & & \\
14 & 0.030 & & 44 & 0.239 & & 74 & 0.698 & & 104 & & \\
15 & 0.104 & & 45 & 0.086 & & 75 & 0.829 & & 105 & & \\
16 & 0.104 & & 46 & 0.207 & & 76 & 0.698 & & 106 & & \\
17 & 0.010 & & 47 & 0.207 & & 77 & 0.798 & & 107 & & \\
18 & 0.104 & & 48 & 0.293 & & 78 & 0.829 & & 108 & & \\
19 & 0.049 & & 49 & 0.293 & & 79 & 0.829 & & 109 & & \\
20 & 0.095 & & 50 & 0.322 & & 80 & 0.916 & & 110 & & \\
21 & 0.010 & & 51 & 0.445 & & 81 & 0.916 & & & & \\
22 & 0.131 & & 52 & 0.322 & & 82 & 0.904 & & & & \\
23 & 0.095 & & 53 & 0.419 & & 83 & 0.916 & & & & \\
24 & 0.113 & & 54 & 0.315 & & 84 & 0.928 & & & & \\
25 & 0.030 & & 55 & 0.507 & & 85 & 1.102 & & & & \\
26 & 0.049 & & 56 & 0.438 & & 86 & 0.900 & & & & \\
27 & 0.122 & & 57 & 0.438 & & 87 & 0.967 & & & & \\
28 & 0.174 & & 58 & 0.464 & & 88 & 0.959 & & & & \\
29 & 0.182 & & 59 & 0.531 & & 89 & 1.001 & & & & \\
30 & 0.182 & & 60 & 0.536 & & 90 & 0.956 & & & & \\
\bottomrule()
\end{longtable}

В рассматриваемой таблице данных присутствует \(n = 110\) наблюдений для
каждой из рассматриваемых переменных.

При условии нормальности данных наши описательные статистики для каждой
переменной выглядят следующим образом:

\[
\overline{x} = \frac{1}{n} \sum \limits_{i=1}^{n} x_i = 
\]

\hypertarget{ux43aux43eux440ux440ux435ux43bux44fux446ux438ux43eux43dux43dux44bux439-ux430ux43dux430ux43bux438ux437-ux447ux438ux441ux43bux43eux432ux44bux445-ux434ux430ux43dux43dux44bux445}{%
\subsection{\texorpdfstring{\textbf{Корреляционный анализ числовых
данных}}{Корреляционный анализ числовых данных}}\label{ux43aux43eux440ux440ux435ux43bux44fux446ux438ux43eux43dux43dux44bux439-ux430ux43dux430ux43bux438ux437-ux447ux438ux441ux43bux43eux432ux44bux445-ux434ux430ux43dux43dux44bux445}}

\hypertarget{ux442ux435ux441ux442-ux433ux435ux442ux435ux440ux43eux441ux43aux435ux434ux430ux441ux442ux438ux447ux43dux43eux441ux442ux438-ux434ux43bux44f-ux440ux44fux434ux430-ux434ux430ux43dux43dux44bux445}{%
\subsection{\texorpdfstring{\textbf{Тест гетероскедастичности для ряда
данных}}{Тест гетероскедастичности для ряда данных}}\label{ux442ux435ux441ux442-ux433ux435ux442ux435ux440ux43eux441ux43aux435ux434ux430ux441ux442ux438ux447ux43dux43eux441ux442ux438-ux434ux43bux44f-ux440ux44fux434ux430-ux434ux430ux43dux43dux44bux445}}

\hypertarget{ux43fux43eux441ux442ux440ux43eux435ux43dux438ux435-ux43bux438ux43dux435ux439ux43dux43eux439-ux43cux43eux434ux435ux43bux438-ux440ux435ux433ux440ux435ux441ux441ux438ux438}{%
\subsection{\texorpdfstring{\textbf{Построение линейной модели
регрессии}}{Построение линейной модели регрессии}}\label{ux43fux43eux441ux442ux440ux43eux435ux43dux438ux435-ux43bux438ux43dux435ux439ux43dux43eux439-ux43cux43eux434ux435ux43bux438-ux440ux435ux433ux440ux435ux441ux441ux438ux438}}

\hypertarget{ux43eux446ux435ux43dux43aux430-ux441ux442ux430ux442ux438ux441ux442ux438ux447ux435ux441ux43aux43eux439-ux437ux43dux430ux447ux438ux43cux43eux441ux442ux438-ux43aux43eux44dux444ux444ux438ux446ux438ux435ux43dux442ux43eux432-ux43bux438ux43dux435ux439ux43dux43eux439-ux43cux43eux434ux435ux43bux438-ux440ux435ux433ux440ux435ux441ux441ux438ux438}{%
\subsection{\texorpdfstring{\textbf{Оценка статистической значимости
коэффициентов линейной модели
регрессии}}{Оценка статистической значимости коэффициентов линейной модели регрессии}}\label{ux43eux446ux435ux43dux43aux430-ux441ux442ux430ux442ux438ux441ux442ux438ux447ux435ux441ux43aux43eux439-ux437ux43dux430ux447ux438ux43cux43eux441ux442ux438-ux43aux43eux44dux444ux444ux438ux446ux438ux435ux43dux442ux43eux432-ux43bux438ux43dux435ux439ux43dux43eux439-ux43cux43eux434ux435ux43bux438-ux440ux435ux433ux440ux435ux441ux441ux438ux438}}

\hypertarget{ux43eux446ux435ux43dux43aux430-ux430ux434ux435ux43aux432ux430ux442ux43dux43eux441ux442ux438-ux43bux438ux43dux435ux439ux43dux43eux439-ux43cux43eux434ux435ux43bux438-ux440ux435ux433ux440ux435ux441ux441ux438ux438}{%
\subsection{\texorpdfstring{\textbf{Оценка адекватности линейной модели
регрессии}}{Оценка адекватности линейной модели регрессии}}\label{ux43eux446ux435ux43dux43aux430-ux430ux434ux435ux43aux432ux430ux442ux43dux43eux441ux442ux438-ux43bux438ux43dux435ux439ux43dux43eux439-ux43cux43eux434ux435ux43bux438-ux440ux435ux433ux440ux435ux441ux441ux438ux438}}

\hypertarget{ux43eux446ux435ux43dux43aux430-ux43fux440ux43eux433ux43dux43eux437ux43dux43eux433ux43e-ux438ux43dux442ux435ux440ux432ux430ux43bux430-ux434ux43bux44f-ux43bux438ux43dux435ux439ux43dux43eux439-ux43cux43eux434ux435ux43bux438-ux440ux435ux433ux440ux435ux441ux441ux438ux438}{%
\subsection{\texorpdfstring{\textbf{Оценка прогнозного интервала для
линейной модели
регрессии}}{Оценка прогнозного интервала для линейной модели регрессии}}\label{ux43eux446ux435ux43dux43aux430-ux43fux440ux43eux433ux43dux43eux437ux43dux43eux433ux43e-ux438ux43dux442ux435ux440ux432ux430ux43bux430-ux434ux43bux44f-ux43bux438ux43dux435ux439ux43dux43eux439-ux43cux43eux434ux435ux43bux438-ux440ux435ux433ux440ux435ux441ux441ux438ux438}}

\hypertarget{ux442ux435ux43cux44b-ux432ux43eux43fux440ux43eux441ux43eux432-ux43dux430-ux437ux430ux449ux438ux442ux443-ux43fux440ux430ux43aux442ux438ux447ux435ux441ux43aux43eux439-ux440ux430ux431ux43eux442ux44b}{%
\section{\texorpdfstring{\textbf{Темы вопросов на защиту практической
работы}}{Темы вопросов на защиту практической работы}}\label{ux442ux435ux43cux44b-ux432ux43eux43fux440ux43eux441ux43eux432-ux43dux430-ux437ux430ux449ux438ux442ux443-ux43fux440ux430ux43aux442ux438ux447ux435ux441ux43aux43eux439-ux440ux430ux431ux43eux442ux44b}}

\begin{enumerate}
\def\labelenumi{\arabic{enumi}.}
\tightlist
\item
  Задачи корреляционного анализа. Выборочный коэффициент линейной
  корреляции (Пирсона) и его свойства. Шкала Чеддока.
\item
  Выборочный коэффициент линейной корреляции (Пирсона) и его свойства.
  Оценка значимости коэффициента корреляции.
\item
  Корреляция и причинная связь. Проблемы корреляционного анализа.
\item
  Ранговая корреляция. Коэффициент ранговой корреляции Спирмена.
\item
  Задачи регрессионного анализа. Функциональная и статистическая связь.
  Аппроксимационные модели. Параметрическое множество функций.
\item
  Линейная регрессия. Определение коэффициентов линейной модели методом
  наименьших квадратов.
\item
  Проверка значимости полученных коэффициентов модели. Проверка
  адекватности модели с помощью критерия Фишера.
\item
  Доверительный интервал прогноза. Проверка адекватности модели с
  помощью критерия Фишера.
\end{enumerate}

\end{document}
