% Options for packages loaded elsewhere
\PassOptionsToPackage{unicode}{hyperref}
\PassOptionsToPackage{hyphens}{url}
%
\documentclass[
]{article}
\usepackage{amsmath,amssymb}
\usepackage{lmodern}
\usepackage{iftex}
\ifPDFTeX
  \usepackage[T1]{fontenc}
  \usepackage[utf8]{inputenc}
  \usepackage{textcomp} % provide euro and other symbols
\else % if luatex or xetex
  \usepackage{unicode-math}
  \defaultfontfeatures{Scale=MatchLowercase}
  \defaultfontfeatures[\rmfamily]{Ligatures=TeX,Scale=1}
  \setmainfont[]{SourceSansPro}
\fi
% Use upquote if available, for straight quotes in verbatim environments
\IfFileExists{upquote.sty}{\usepackage{upquote}}{}
\IfFileExists{microtype.sty}{% use microtype if available
  \usepackage[]{microtype}
  \UseMicrotypeSet[protrusion]{basicmath} % disable protrusion for tt fonts
}{}
\makeatletter
\@ifundefined{KOMAClassName}{% if non-KOMA class
  \IfFileExists{parskip.sty}{%
    \usepackage{parskip}
  }{% else
    \setlength{\parindent}{0pt}
    \setlength{\parskip}{6pt plus 2pt minus 1pt}}
}{% if KOMA class
  \KOMAoptions{parskip=half}}
\makeatother
\usepackage{xcolor}
\usepackage[margin=1in]{geometry}
\usepackage{graphicx}
\makeatletter
\def\maxwidth{\ifdim\Gin@nat@width>\linewidth\linewidth\else\Gin@nat@width\fi}
\def\maxheight{\ifdim\Gin@nat@height>\textheight\textheight\else\Gin@nat@height\fi}
\makeatother
% Scale images if necessary, so that they will not overflow the page
% margins by default, and it is still possible to overwrite the defaults
% using explicit options in \includegraphics[width, height, ...]{}
\setkeys{Gin}{width=\maxwidth,height=\maxheight,keepaspectratio}
% Set default figure placement to htbp
\makeatletter
\def\fps@figure{htbp}
\makeatother
\setlength{\emergencystretch}{3em} % prevent overfull lines
\providecommand{\tightlist}{%
  \setlength{\itemsep}{0pt}\setlength{\parskip}{0pt}}
\setcounter{secnumdepth}{-\maxdimen} % remove section numbering
\ifLuaTeX
  \usepackage{selnolig}  % disable illegal ligatures
\fi
\IfFileExists{bookmark.sty}{\usepackage{bookmark}}{\usepackage{hyperref}}
\IfFileExists{xurl.sty}{\usepackage{xurl}}{} % add URL line breaks if available
\urlstyle{same} % disable monospaced font for URLs
\hypersetup{
  pdftitle={Практическая работа 4. Генерация распределений. Проверка определений известных распределений},
  pdfauthor={Юрченков Иван Александрович, ассистент кафедры ПМ},
  hidelinks,
  pdfcreator={LaTeX via pandoc}}

\title{Практическая работа 4. Генерация распределений. Проверка
определений известных распределений}
\author{Юрченков Иван Александрович, ассистент кафедры ПМ}
\date{2022-10-07}

\begin{document}
\maketitle

\hypertarget{ux43fux43eux441ux442ux430ux43dux43eux432ux43aux430-ux437ux430ux434ux430ux447ux438}{%
\section{\texorpdfstring{\textbf{Постановка
задачи}}{Постановка задачи}}\label{ux43fux43eux441ux442ux430ux43dux43eux432ux43aux430-ux437ux430ux434ux430ux447ux438}}

\begin{enumerate}
\def\labelenumi{\arabic{enumi}.}
\tightlist
\item
  \textbf{Сгенерировать выборку нормального распределения
  \(\ Y\sim N(\mu, \sigma^2)\ \) используя определение центральной
  предельной теоремы}.
\end{enumerate}

\begin{itemize}
\tightlist
\item
  На основе \(\ n \approx 10\div 20\ \) равномерно распределенных
  случайных реализаций случайных величин образовать новую выборку по
  определению центральной предельной теоремы.
\end{itemize}

Если \(\ Y_{i} \sim U(a_i, b_i), \ i = 1, 2, \dots, n\ \), где
\(\ Y_{i}\ -\) равномерно распределенная реализация случайной величины
со случайными параметрами
\(\ a_i \in \mathbb{R},\ b_i \in \mathbb{R}\ \), то ожидаемая нормально
распределенная величина \(\ Y\ \) будет найдена как:

\[
Y = \sum_{i=1}^{n} Y_i, \ i=1,2,\dots, n
\]

\begin{itemize}
\item
  Для получившейся выборки построить гистограмму, визуализировать на
  гистограмме теоретическую плотность нормального распределения по
  несмещенным точечным оценкам \(\ \hat{\mu}, \hat{\sigma}\ \).
\item
  Провести тест на нормальное распределение с помощью критерия
  \(\ \chi^2\)-Пирсона. Степени свободы рассчитывать как
  \(\ k = n - 1\ \).
\end{itemize}

Для генерации выборок рекомендуется пользоваться встроенными в
компьютерные статистические пакеты функциями генерации
\textbf{равномерно распределённых случайных величин}, которые задаются с
помощью параметров границ интервала генерации чисел \(a\) и \(b\).

\begin{enumerate}
\def\labelenumi{\arabic{enumi}.}
\setcounter{enumi}{1}
\tightlist
\item
  \textbf{Сгенерировать выборку \(\ \chi^2\)-распределения
  \(\ R \sim \chi_{k}^2\ \) используя определение распределения
  \(\ \chi^2\ \)}.
\end{enumerate}

\begin{itemize}
\tightlist
\item
  На основе \(\ Z\)-оценок нормально распределенных случайных реализаций
  случайных величин \(\ L_{i} \sim N(\mu_i, \sigma_i^2)\ \) образовать
  новую выборку по определению \(\ \chi^2\)-распределения:
\end{itemize}

\[
R = \sum_{i = 1}^{n} Z[L_i]^2,\ \ Z[L_i] = \frac{L - E[L]}{\sigma[L]}, \ L_i \sim N(\mu_i, \sigma_i^2),\ i = 1,2,\dots,n
\]

\begin{itemize}
\item
  Для получившейся выборки построить гистограмму, визуализировать на
  гистограмме теоретическую плотность \(\ \chi_k^2\ \) распределения c
  \(\ k = n-1\ \) степенями свободы.
\item
  Провести тест на \(\ \chi^2\ \) с помощью критерия
  \(\ \chi^2\)-Пирсона.
\end{itemize}

Для генерации \textbf{нормально распределенных реализаций} случайных
величин рекомендуется пользоваться встроенными в статистические пакеты
функциями для генерации значений выборки из нормального распределения,
которые задаются с помощью параметров математического ожидания \(\mu\) и
стандатрного отклонения \(\sigma^2\).

\begin{enumerate}
\def\labelenumi{\arabic{enumi}.}
\setcounter{enumi}{2}
\tightlist
\item
  \textbf{Сгенерировать выборку распределения Фишера на основе
  определения}.
\end{enumerate}

\begin{itemize}
\tightlist
\item
  На основе двух случайных реализаций \(\ Y_1, Y_2\ \) случайных
  величин, распределенных по \(\chi^2\)-распределению со степенями
  свободы \(\ d1, d2\ \) соответственно, сгенерировать выборку,
  распределенную по распределению Фишера \(\ S \sim F(d1, d2)\ \) в
  соответствии с определением:
\end{itemize}

\[
S = \frac{Y_1 / d_1}{Y_2 / d_2}, \ S\sim F(d_1, d_2).
\]

\begin{itemize}
\item
  Для получившейся выборки построить гистограмму, визуализировать на
  гистограмме теоретическую плотность \(\ F(d1, d2)\ \) распределения.
\item
  Провести тест на распределение Фишера с помощью критерия
  \(\ \chi^2\)-Пирсона.
\end{itemize}

Для генерации выборки фиксированного размера из распределения \(\chi^2\)
рекомендуется пользоваться встроенными в статистические пакеты функциями
для генерации случайных выборок из распределения \(\chi^2\) с \(df\)
степенями свободы.

\begin{enumerate}
\def\labelenumi{\arabic{enumi}.}
\setcounter{enumi}{3}
\tightlist
\item
  \textbf{Сгенерировать выборку t-распределения на основе определения}.
\end{enumerate}

\begin{itemize}
\tightlist
\item
  На основе \(\ n \approx 2\div 8\ \) случайных реализаций
  \(\ Y_1, Y_2, \dots, Y_n\ \) случайных величин, распределенных по
  стандартному нормальному распределению
  \(\ Y_i \sim N(0, 1), \ i = 1, 2, \dots, n\ \), сгенерировать выборку
  \(\ T \sim t(n)\ \), распределенную по \(t\)-распределению Стьюдента с
  \(\ df = n\ \) степенями свободы в соответствии с определением:
\end{itemize}

\[
T = \frac{Y_0}{\sqrt{\frac{1}{n} \sum_{i=1}^{n} Y_i^2}}, \quad Y_0 \sim N(0, 1).
\]

\begin{itemize}
\tightlist
\item
  Реализовать вычисление аналитической плотности \(t\)-распределения
  Стьюдента с использованием бета-функции:
\end{itemize}

\[
{\displaystyle p_{t}(x\ |\ n)={\frac {1}{{\sqrt {n}}\,\mathrm {B} ({\frac {1}{2}},{\frac {n}{2}})}}\left(1+{\frac {x^{2}}{n}}\right)^{\!-{\frac {n+1}{2}}}},
\] где

\[
{\displaystyle \mathrm {B} (x,y)=\int \limits _{0}^{1}t^{x-1}(1-t)^{y-1}\,dt,}
\]

определённая при \({\displaystyle \operatorname {Re} x>0}\),
\({\displaystyle \operatorname {Re} y>0}\).

\begin{itemize}
\item
  Для получившейся выборки построить гистограмму, визуализировать на
  гистограмме теоретическую плотность \(\ t(n)\ \).
\item
  Для получившейся выбрки провести тест на \(t\)-распределение Стьюдента
  с помощью критерия \(\ \chi^2\)-Пирсона, используя в качестве функции
  вероятности распределения \(\ P_t(x\ |\ n)\):
\end{itemize}

\[
P_t(x\ |\ n) = \int_{-\infty}^{x} p_t(z\ |\ n) dz.
\]

\begin{enumerate}
\def\labelenumi{\arabic{enumi}.}
\setcounter{enumi}{4}
\tightlist
\item
  Для всех заданий количество генерируемых значений выборки установить
  равным \(N \approx 100 \div 1000\). Уровень надежности для критерия
  \(\chi^2\)-Пирсона или метода анаморфоз \(\gamma = 0.95\).
\end{enumerate}

\hypertarget{ux432ux43eux43fux440ux43eux441ux44b-ux43dux430-ux437ux430ux449ux438ux442ux443-ux43fux440ux430ux43aux442ux438ux447ux435ux441ux43aux43eux439-ux440ux430ux431ux43eux442ux44b}{%
\section{\texorpdfstring{\textbf{Вопросы на защиту практической
работы}}{Вопросы на защиту практической работы}}\label{ux432ux43eux43fux440ux43eux441ux44b-ux43dux430-ux437ux430ux449ux438ux442ux443-ux43fux440ux430ux43aux442ux438ux447ux435ux441ux43aux43eux439-ux440ux430ux431ux43eux442ux44b}}

\begin{enumerate}
\def\labelenumi{\arabic{enumi}.}
\item
  Центральная предельная теорема. Реализации случайно распределенных
  величин. Независимые величины. Степени свободы суммы независимо
  распределенных величин.
\item
  Определение нормального распределения. Спрямление для координат
  нормального распределения. Определение параметров нормального
  распределения через точечные оценки. Определение параметров
  нормального распределения, образованного суммой независимых величин,
  через ЦПТ.
\item
  Определение распределения Фишера. Аналитические формулы
  математического ожидания и дисперсии расределения Фишера.
\item
  t-распределение Стьюдента. Аппроксимации и определение функции
  плотности. Смесь нормально расределенных величин. Определение
  \(Z\)-оценок.
\end{enumerate}

\end{document}
