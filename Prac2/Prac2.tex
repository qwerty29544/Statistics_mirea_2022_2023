% Options for packages loaded elsewhere
\PassOptionsToPackage{unicode}{hyperref}
\PassOptionsToPackage{hyphens}{url}
%
\documentclass[
]{article}
\usepackage{amsmath,amssymb}
\usepackage{lmodern}
\usepackage{iftex}
\ifPDFTeX
  \usepackage[T1]{fontenc}
  \usepackage[utf8]{inputenc}
  \usepackage{textcomp} % provide euro and other symbols
\else % if luatex or xetex
  \usepackage{unicode-math}
  \defaultfontfeatures{Scale=MatchLowercase}
  \defaultfontfeatures[\rmfamily]{Ligatures=TeX,Scale=1}
  \setmainfont[]{SourceSansPro}
\fi
% Use upquote if available, for straight quotes in verbatim environments
\IfFileExists{upquote.sty}{\usepackage{upquote}}{}
\IfFileExists{microtype.sty}{% use microtype if available
  \usepackage[]{microtype}
  \UseMicrotypeSet[protrusion]{basicmath} % disable protrusion for tt fonts
}{}
\makeatletter
\@ifundefined{KOMAClassName}{% if non-KOMA class
  \IfFileExists{parskip.sty}{%
    \usepackage{parskip}
  }{% else
    \setlength{\parindent}{0pt}
    \setlength{\parskip}{6pt plus 2pt minus 1pt}}
}{% if KOMA class
  \KOMAoptions{parskip=half}}
\makeatother
\usepackage{xcolor}
\usepackage[margin=1in]{geometry}
\usepackage{graphicx}
\makeatletter
\def\maxwidth{\ifdim\Gin@nat@width>\linewidth\linewidth\else\Gin@nat@width\fi}
\def\maxheight{\ifdim\Gin@nat@height>\textheight\textheight\else\Gin@nat@height\fi}
\makeatother
% Scale images if necessary, so that they will not overflow the page
% margins by default, and it is still possible to overwrite the defaults
% using explicit options in \includegraphics[width, height, ...]{}
\setkeys{Gin}{width=\maxwidth,height=\maxheight,keepaspectratio}
% Set default figure placement to htbp
\makeatletter
\def\fps@figure{htbp}
\makeatother
\setlength{\emergencystretch}{3em} % prevent overfull lines
\providecommand{\tightlist}{%
  \setlength{\itemsep}{0pt}\setlength{\parskip}{0pt}}
\setcounter{secnumdepth}{-\maxdimen} % remove section numbering
\ifLuaTeX
  \usepackage{selnolig}  % disable illegal ligatures
\fi
\IfFileExists{bookmark.sty}{\usepackage{bookmark}}{\usepackage{hyperref}}
\IfFileExists{xurl.sty}{\usepackage{xurl}}{} % add URL line breaks if available
\urlstyle{same} % disable monospaced font for URLs
\hypersetup{
  pdftitle={Практическое занятие №2. Интервальные оценки параметров. Доверительные интервалы точечных оценок},
  pdfauthor={Юрченков Иван Александрович, ассистент кафедры ПМ},
  hidelinks,
  pdfcreator={LaTeX via pandoc}}

\title{Практическое занятие №2. Интервальные оценки параметров.
Доверительные интервалы точечных оценок}
\author{Юрченков Иван Александрович, ассистент кафедры ПМ}
\date{2022-09-11}

\begin{document}
\maketitle

\hypertarget{ux43fux43eux441ux442ux430ux43dux43eux432ux43aux430-ux437ux430ux434ux430ux447ux438}{%
\section{\texorpdfstring{\textbf{Постановка
задачи}}{Постановка задачи}}\label{ux43fux43eux441ux442ux430ux43dux43eux432ux43aux430-ux437ux430ux434ux430ux447ux438}}

\begin{enumerate}
\def\labelenumi{\arabic{enumi}.}
\item
  Скачать папку с исходными данными по
  \href{https://disk.yandex.ru/d/PwFd-L8zn7x8eQ}{ссылке}
\item
  Открыть папку соответствующую номеру своей группы
\item
  Открыть папку соответствующую номеру своего варианта
\item
  В папке \textbf{data} можете найти 4 ряда данных реализации случайной
  величины
\item
  Для каждого из четырех рядов данных необходимо провести следующие
  расчёты:
\end{enumerate}

\begin{itemize}
\tightlist
\item
  Подсчитать выборочные статистики для среднего и стандартного
  отклонения по следующим формулам:
\end{itemize}

Выборочное среднее:

\[
  \overline{X_{B}} = \frac{1}{N} \cdot \sum_{i=1}^{N} x_i,
\] где \(N \ -\) число значений реализации случайной величины
(количество значений в ряде данных),

\(x_i \in \mathbb{R}, i \in \overline{1, N} \ -\) реализации нашей
случайной величины (значения ряда данных).

~

Выборочное среднеквадратическое отклонение:

\[
\sigma_{B} = \sqrt{\frac{\sum_{i=1}^{N}\left( x_i - \overline{X_{B}}  \right)^2}{N-1}}.
\]

\begin{itemize}
\tightlist
\item
  Для выборочного среднего \(\overline{X_{B}}\) подсчитать границы
  доверительного интервала по правилу нормального распределния,
  используя таблицу критических значений функции Лапласа \(\Phi(x)\):
\end{itemize}

\[
\hat{X}_{B} \in \left[ \overline{X_{B}} - X_{\gamma} \cdot \frac{\sigma_{B}}{\sqrt{N}}, \overline{X_{B}} + X_{\gamma} \cdot \frac{\sigma_{B}}{\sqrt{N}} \right], \quad \Phi(X_{\gamma}) = \frac{\gamma}{2}
\]

~

и по правилу t-распределения Стьюдента используя таблицу критических
значений \(t_{\gamma, n}\) t-распределения:

\[
\hat{X}_{B} \in \left[ \overline{X_{B}} - t_{1- \gamma, N-1} \cdot \frac{\sigma_{B}}{\sqrt{N}}, \overline{X_{B}} + t_{1-\gamma, N-1} \cdot \frac{\sigma_{B}}{\sqrt{N}} \right]
\]

при значении уверенности \(\gamma = 0.95\)

\begin{itemize}
\tightlist
\item
  Для выборочного среднеквадратического отклонения \(\sigma_{B}\)
  подсчитать границы доверительного интервала по оценке
  \(\chi^2\)-распределения при значении уверенности \(\gamma\) = 0.95:
\end{itemize}

\[
\hat{\sigma}_{B} \in \left[ \frac{\sigma_{B}\cdot \sqrt{N-1}}{\sqrt{\chi_{\frac{1+\gamma}{2},N-1}^2}},  \frac{\sigma_{B} \cdot \sqrt{N-1}}{\sqrt{\chi_{\frac{1-\gamma}{2}, N-1}^2}} \right]
\]

\hypertarget{ux43fux440ux438ux43cux435ux440-ux440ux430ux441ux447ux435ux442ux430}{%
\section{\texorpdfstring{\textbf{Пример
расчета}}{Пример расчета}}\label{ux43fux440ux438ux43cux435ux440-ux440ux430ux441ux447ux435ux442ux430}}

\hypertarget{ux434ux430ux43dux43e}{%
\subsection{\texorpdfstring{\textbf{Дано}}{Дано}}\label{ux434ux430ux43dux43e}}

Дан вещественный ряд данных реализации случайной величины:

\[
X = \begin{matrix}
 (1.63, & 1.80, & 1.69, & 1.73, & 1.79, & 1.76\ , \\
\ 1.77, & 1.91, & 1.70, & 1.69, & 1.50, & 1.94\ , \\ 
\ 1.90, & 1.61, & 2.10, & 1.62, & 1.79, & 1.58\ , \\
\ 1.81, & 1.69, & 1.61, & 1.68, & 1.72, & 1.84\ , \\
\ 1.66, & 1.86, & 1.57, & 1.71, & 1.58, & 1.68)
\end{matrix}.
\]

Для данного ряда \(N = 30\).

\hypertarget{ux440ux430ux441ux447ux435ux442-ux432ux44bux431ux43eux440ux43eux447ux43dux44bux445-ux441ux442ux430ux442ux438ux441ux442ux438ux43a}{%
\subsection{\texorpdfstring{\textbf{Расчет выборочных
статистик}}{Расчет выборочных статистик}}\label{ux440ux430ux441ux447ux435ux442-ux432ux44bux431ux43eux440ux43eux447ux43dux44bux445-ux441ux442ux430ux442ux438ux441ux442ux438ux43a}}

Для ряда данных \(X\) расчитаем выборочное среднее \(\overline{X_{B}}\):

\(\overline{X_{B}} = \frac{1}{N} \cdot \sum_{i=1}^{N} x_i = \frac{1.63 + 1.8 + 1.69 + \dots + 1.58 + 1.68}{30} \approx\)
1.731.

Для ряда данных \(X\) расчитаем выборочное СКО \({\sigma_{B}}\):

\(\sigma_{B} = \sqrt{\frac{\sum_{i=1}^{N}\left( x_i - \overline{X_{B}} \right)^2}{N-1}} = \sqrt{\frac{(1.731 - 1.63)^2 + (1.731 - 1.8)^2 + \dots + (1.731 - 1.68)^2}{30-1}} \approx\)
0.129.

\hypertarget{ux440ux430ux441ux447ux435ux442-ux434ux43eux432ux435ux440ux438ux442ux435ux43bux44cux43dux43eux433ux43e-ux438ux43dux442ux435ux440ux432ux430ux43bux430-ux43dux43eux440ux43cux430ux43bux44cux43dux43eux433ux43e-ux440ux430ux441ux43fux440ux435ux434ux435ux43bux435ux43dux438ux44f}{%
\subsection{\texorpdfstring{\textbf{Расчет доверительного интервала
нормального
распределения}}{Расчет доверительного интервала нормального распределения}}\label{ux440ux430ux441ux447ux435ux442-ux434ux43eux432ux435ux440ux438ux442ux435ux43bux44cux43dux43eux433ux43e-ux438ux43dux442ux435ux440ux432ux430ux43bux430-ux43dux43eux440ux43cux430ux43bux44cux43dux43eux433ux43e-ux440ux430ux441ux43fux440ux435ux434ux435ux43bux435ux43dux438ux44f}}

По полученным \(\overline{X_{B}}\) и \(\sigma_{B}\) получим следующий
доверительный интервал точечной оценки \(\hat{X}_{B}\) со значением
уверенности \(\gamma = 0.95\):

\[
\hat{X}_{B} = \hat{X}_{B} \in \left[ 1.731 - X_{0.95} \cdot \frac{0.129}{\sqrt{30}}, 1.731 + X_{0.95} \cdot \frac{0.129}{\sqrt{30}} \right], \quad \Phi(X_{0.95}) = \frac{0.95}{2} = 0.475
\]

По таблице критических значений функции Лапласа значение
\(X_{0.95} = 1.96\). Следовательно доверительный интервал расчитывается
следующим образом:

\[
\hat{X}_{B} = \hat{X}_{B} \in \left[ 1.731 - 1.96 \cdot \frac{0.129}{\sqrt{30}}, 1.731 + 1.96 \cdot \frac{0.129}{\sqrt{30}} \right], \quad \Phi(X_{0.95}) = \frac{0.95}{2} = 0.475.
\]

И доверительный интервал равен:

\[
\hat{X}_{B} = \hat{X}_{B} \in \left[ 1.731 - 1.96 \cdot \frac{0.129}{\sqrt{30}}, 1.731 + 1.96 \cdot \frac{0.129}{\sqrt{30}} \right],
\]

или

\[
\hat{X}_{B} = \hat{X}_{B} \in \left[ 1.685,\, 1.777 \right].
\]

\hypertarget{ux440ux430ux441ux447ux435ux442-ux434ux43eux432ux435ux440ux438ux442ux435ux43bux44cux43dux43eux433ux43e-ux438ux43dux442ux435ux440ux432ux430ux43bux430-ux43fux43e-t-ux440ux430ux441ux43fux440ux435ux434ux435ux43bux435ux43dux438ux44e-ux441ux442ux44cux44eux434ux435ux43dux442ux430}{%
\subsection{\texorpdfstring{\textbf{Расчет доверительного интервала по
t-распределению
Стьюдента}}{Расчет доверительного интервала по t-распределению Стьюдента}}\label{ux440ux430ux441ux447ux435ux442-ux434ux43eux432ux435ux440ux438ux442ux435ux43bux44cux43dux43eux433ux43e-ux438ux43dux442ux435ux440ux432ux430ux43bux430-ux43fux43e-t-ux440ux430ux441ux43fux440ux435ux434ux435ux43bux435ux43dux438ux44e-ux441ux442ux44cux44eux434ux435ux43dux442ux430}}

По полученным \(\overline{X_{B}}\) и \(\sigma_{B}\) получим следующий
доверительный интервал точечной оценки \(\hat{X}_{B}\) со значением
уверенности \(\gamma = 0.95\):

\[
\hat{X}_{B} = \hat{X}_{B} \in \left[ 1.731 - t_{0.05, 29} \cdot \frac{0.129}{\sqrt{30}}, 1.731 + t_{0.05, 29} \cdot \frac{0.129}{\sqrt{30}} \right].
\]

По таблице критических значений t-распределения Стьюдента значение
\(t_{0.05, 29} = 2.05\). Следовательно доверительный интервал
расчитывается следующим образом:

\[
\hat{X}_{B} = \hat{X}_{B} \in \left[ 1.731 - 2.05 \cdot \frac{0.129}{\sqrt{30}}, 1.731 + 2.05 \cdot \frac{0.129}{\sqrt{30}} \right].
\]

Итоговый доверительный интервал для точечной оценки выборочного среднего
равен:

\[
\hat{X}_{B} = \hat{X}_{B} \in \left[ 1.683,\, 1.779 \right].
\]

Доверительный интервал для точечной оценки выборочного среднего по
t-распределению Стьюдента оказался шире чем интервал расчитаный по
нормальному расределению, что является ожидаемым результатом поскольку
распределение Стьюдента является более пологим в хвостах при малых
значениях степеней свободы реализации случайной величины.

\hypertarget{ux440ux430ux441ux447ux435ux442-ux434ux43eux432ux435ux440ux438ux442ux435ux43bux44cux43dux43eux433ux43e-ux438ux43dux442ux435ux440ux432ux430ux43bux430-ux441ux440ux435ux434ux43dux435ux433ux43e-ux43aux432ux430ux434ux440ux430ux442ux430-ux43eux442ux43aux43bux43eux43dux435ux43dux438ux44f}{%
\subsection{\texorpdfstring{\textbf{Расчет доверительного интервала
среднего квадрата
отклонения}}{Расчет доверительного интервала среднего квадрата отклонения}}\label{ux440ux430ux441ux447ux435ux442-ux434ux43eux432ux435ux440ux438ux442ux435ux43bux44cux43dux43eux433ux43e-ux438ux43dux442ux435ux440ux432ux430ux43bux430-ux441ux440ux435ux434ux43dux435ux433ux43e-ux43aux432ux430ux434ux440ux430ux442ux430-ux43eux442ux43aux43bux43eux43dux435ux43dux438ux44f}}

Точечные оценки выборочной дисперсии являются распределенными по
\(\chi^2\)-распределению, что делает возможным оценивать с помощью
квантилей \(\alpha_{1, 2}=\frac{1 \pm \gamma}{2}\) доверительные
интервалы как для точечной оценки дисперсии выборки с малыми степенями
свободы, так и для точечной оценки СКО.

Доверительный интервал точечной оценки \(\hat{\sigma}_{B}\) СКО по
известным \(\sigma_{B} = 0.129\) и \(N = 30\) при \(\gamma = 0.95\)
расчитывается следующим образом:

\[
\hat{\sigma}_{B} \in \left[ \frac{0.129 \cdot \sqrt{29}}{\sqrt{\chi_{0.975, 29}^2}},  \frac{0.129 \cdot \sqrt{29}}{\sqrt{\chi_{0.025, 29}^2}} \right].
\]

Из таблицы квантилей \(\chi^2\)-распределения найдем значения для
\(\chi_{0.975, 29}^2\) и \(\chi_{0.025, 29}^2\):

\[
\chi_{0.975, 29}^2 = 45.7, \quad \chi_{0.025, 29}^2 = 16.0,
\]

и доверительный интервал для точечной оценки СКО равен:

\[
\hat{\sigma}_{B} \in \left[ \frac{0.129 \cdot \sqrt{29}}{\sqrt{45.7}},  \frac{0.129 \cdot \sqrt{29}}{\sqrt{16.0}} \right].
\]

\[
\hat{\sigma}_{B} \in \left[ 0.103,\, 0.174 \right].
\]

\hypertarget{ux442ux435ux43cux44b-ux432ux43eux43fux440ux43eux441ux43eux432-ux43dux430-ux437ux430ux449ux438ux442ux443-ux43fux440ux430ux43aux442ux438ux447ux435ux441ux43aux43eux439-ux440ux430ux431ux43eux442ux44b}{%
\section{\texorpdfstring{\textbf{Темы вопросов на защиту практической
работы}}{Темы вопросов на защиту практической работы}}\label{ux442ux435ux43cux44b-ux432ux43eux43fux440ux43eux441ux43eux432-ux43dux430-ux437ux430ux449ux438ux442ux443-ux43fux440ux430ux43aux442ux438ux447ux435ux441ux43aux43eux439-ux440ux430ux431ux43eux442ux44b}}

\begin{enumerate}
\def\labelenumi{\arabic{enumi}.}
\item
  Определение состоятельности, смещенности и эффективности для точечных
  оценок параметров выборки.
\item
  Понятие квантилей, квартилей и децилей в исследовании доверительных
  интервалов точечных оценок параметров. Распределение точечных оценок
  параметров.
\item
  \(Z\)-оценки и предпосылки для определения доверительных интервалов
  точечных оценок на основе нормального распределения.
\item
  Использование \(t\)-распределения Стьюдента в определении
  доверительных интервалов точечных оценок. Различия между
  \(t\)-распределением и нормальным распределением.
\item
  Предпосылки использования \(\chi^2\)-распределения в определении
  доверительных интервалов точечных оценок выборочного СКО.
\end{enumerate}

\end{document}
